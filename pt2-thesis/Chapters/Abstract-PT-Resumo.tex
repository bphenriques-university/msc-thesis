\noindent Os agentes autónomos estão cada vez mais a ter um lugar na nossa sociedade em áreas como fisioterapia, entretenimento, e até tratamento de obesidade. No entanto, não é suficiente criar agents capazes de criar boas primeiras impressões. É necessário que estes agentes sejam capazes de produzir este efeito continuamente para que as pessoas se sintam encorajadas a interagir frequentemente com o agente. Por outras palavras, o agente tem que ser capaz de construir e manter rapport mostrando sinais de positividade, atenção mútua e de coordenação; por exemplo, motivar, estabelecer contacto visual e espelhar a postura, respectivamente. Actualmente, existem poucos agentes capazes de demonstrar estes tipos de sinais simultaneamente e, aqueles que o conseguem, não são robóticos. De modo a colmatar esta lacuna, construímos um modelo de rapport para que agentes, robóticos ou virtuais, possam produzir naturalmente estes sinais e consigam adaptar-se continuamente à interação. O modelo foi implementado usando a \acf{SERA} framework e testado com o robot \acf{EMYS}, num cenário denominado Quick Numbers, onde nos foi possível verificar a agradabilidade, inteligência, naturalidade e a proximidade do agente, tal como percepcionada pelos participantes. Os resultados não foram estatisticamente significativos, no entanto, por análise das gravações de vídeo, verificou-se uma maior frequência de reações positivas pelos participantes na condição de rapport, pelo que, uma amostragem superior à efectuada ($n=20$) poderá revelar os resultados estatísticos esperados. Por fim, com pouco esforço, o modelo pode ser integrado em qualquer agente que utilize a \ac{SERA} framework.