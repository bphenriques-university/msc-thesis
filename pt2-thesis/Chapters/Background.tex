\fancychapter{Background: Rapport}
\label{chap:rapport}

%The current chapter introduces the concept of rapport since it is the main phenomenon we want to elicit on people's perception. 
%\section*{\centering *}

The feeling of flow and connection during interactions is formally known as rapport~\cite{Wang2009} that affects people on three levels: emotional, behavioural and cognitive~\cite{Wang2009}.

The emotional level refers to the impact the relationship has on partners, while behavioural alludes to, for example, the convergence of movements and facial expressions. Finally, the cognitive level refers to a shared understanding between conversational partners~\cite{Wang2009}.

Spencer Oatey suggests that rapport management can be divided into three main objectives: enhance, maintain or destroy~\cite{Spencer-Oatey2005}. The first objective aims to create strong first impressions, the second encourages the continuation and the third focus on destroying relationships. Each one of these goals can only be satisfied by interacting with the external world~\cite{Papangelis2014, Zhao2014} (Figure~\ref{table:BuildingRapportPlan}). Each one of these objectives manipulates differently the three components of rapport suggested by Tickle-Degnen and Rosenthal~\cite{Tickle-Degnen1990}:

\begin{itemize}
	\item \textbf{Positivity}: the feeling of approval and friendliness (e.g. humour and self-disclosure);
	\item \textbf{Mutual attention}: the feeling that the other's attention is focused on the individual (e.g. mutual gaze and ``Hmm hmmm'' vocalisations);
	\item \textbf{Coordination}: the feeling of predictability and being in-sync (e.g. respect social norms and mimic posture).
\end{itemize}

However, in order to build rapport, it is essential that the three components above mentioned co-exist during interactions~\cite{Grahe1999, Wang2010, Zhao2014, Cassell2007}. For example, using gaze to establish mutual attention might convey disinterest (and have negative social effect) if not accompanied by other behaviours that stimulate positivity and coordination~\cite{Wang2010}. Although, the relative weights of these components may change as the relationship evolves beyond strangers~\cite{Wang2010, Zhao2014, Cassell2007}.

\begin{figure}[H]
	\centering
	%\begin{framed}
		\scalebox{0.6}{
			\begin{forest}
				[\textbf{Build Rapport}
				[Stimulate Positivity 
					[Elicit Positive Emotions [Smile] [Embarrassed Laugh][Humour]]
					[Self-disclosure]]
				[Stimulate Mutual Attention
					[Mutual Gaze] [Backchannel][Paraphrasing]]
				[Stimulate Coordination
					[Backchannel]
					[Postural Mirror]
					[Adhere To Social Norms
							[Politeness] [Respect Personal Space]]]
				]
			\end{forest}
		}
	%\end{framed}
	\caption{Example plan for building rapport. The nodes are goals and the leafs are actions.}
	\label{table:BuildingRapportPlan}
\end{figure}

Moreover, two strangers behaviours are initially driven by cultural conventions as they do not know each other and, therefore, they expect what was taught by their cultural environment according to the current context~\cite{Zhao2014} (e.g. greet and be polite). As the relationship evolves and the interlocutors get to know one another, the weight of positivity decreases, while coordination increases, and mutual attention remains constant~\cite{Zhao2014, Tickle-Degnen1990}. The interlocutors may even violate what is culturally accepted in order to meet interactional goals and behavioural expectations~\cite{Zhao2014}. For example, friends may use sarcasm and insults instead of politeness~\cite{Zhao2014}. Table~\ref{table:rapportStrategies} describes examples of verbal and non-verbal strategies for managing rapport.

As the relationship evolves and the interlocutors get to know one another, the weight of positivity decreases, coordination increases and mutual attention remains constant 

\begin{table}[H]
	\centering
	\begin{tabular}{|l|l|}
	\hline
	\textbf{Verbal} & \specialcell{Humour; Small-talk; Paraphrasing; Self disclosure; Praise; Ego Suspension;\\ Refer to shared experience; Slower rate of speech} \\ \hline
	\textbf{Non-Verbal} & Mutual Gaze; Smile; Reciprocate previous action; Silence; Postural mimicry \\ \hline
	\end{tabular}
	\caption{Examples of verbal and non-verbal strategies to manage rapport.}
	\label{table:rapportStrategies}
\end{table}

Learning and set behavioural expectation is also important to manage rapport~\cite{Nomura2015} in order to tailor strategies to the current nature of the relationship that is continuously in evolution. This can be achieved through the use of self-disclosure \cite{Moon2000} (share information about oneself), small-talk~\cite{Cassell2003} (a conversation that does not cover any functional topic) and humour~\cite{Treger2013} (provoke laughter). For instance, when self-disclosure is successful, it is reciprocal, intimacy increases, disclosed topics become more diversified, and deeper~\cite{Zhao2014}. Moreover, assessing when rapport strategies were successful is also important, for example, mutual gaze and smiling become more noticeable and consistent\cite{Grahe1999, Zhao2014}.

To sum up, rapport is a phenomenon that makes interactions more engaging and harmonious. Rapport management can be modelled as a problem of goal satisfaction that is solved through the realisation of actions. In addition, the strategies for managing rapport must take into account the goals of the interaction and the sociocultural context of the interaction in order to satisfy behavioural expectations. Nevertheless, these expectations may not be clear at first and strategies for learning them should be applied.

