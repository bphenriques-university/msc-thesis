% https://www.lightbluetouchpaper.org/2007/03/14/how-not-to-write-an-abstract/
% http://web.ece.ucdavis.edu/~jowens/biberrors.html

\subsection{Overall Discussion}
\label{subsec:RelWorkDiscussion}

Developing a computational models capable of managing rapport similarly to humans is not an easy feat. Researchers had to focus their research on different aspects of rapport and assess their overall contribution. Table~\ref{fig:comparison:rapportSystems} and Table~\ref{fig:comparison:vh:systems}, respectively, compares the systems regarding how they learn social behaviours, and the used rapport management strategies.

\addtolength{\tabcolsep}{1pt}
\begin{table}
	\centering
	\begin{tabular}{lccccccccccc}
		& \textbf{Type}
		& \textbf{Agent} 
	  	& \rot[70]{\textbf{Gaze}}
	  	& \rot[70]{\textbf{Backchannel}} 
	  	& \rot[70]{\textbf{Small-Talk}} 
	  	& \rot[70]{\textbf{Facial Expressions}}
	  	& \rot[70]{\textbf{Gestures}} 
	  	& \rot[70]{\textbf{Mirroring}} 
	  	& \rot[70]{\textbf{Smile}}
	  	& \rot[70]{\textbf{Turn Taking}}
	  	& \rot[70]{\textbf{Praise}}
	  	\\
	  	\midrule
	  	%%%%%%%%%%%%%%%%%%%%%%%%%%%%%%%%%%%%%%%%%%%%%%%%%%%%%%%%%%%%%%
	  	Mutlu et al.~\cite{Mutlu2006} & Rule-based & Robotic & \cmark & \xmark & \xmark & \xmark & \xmark & \xmark & \xmark & \xmark & \xmark\\
	  	%%%%%%%%%%%%%%%%%%%%%%%%%%%%%%%%%%%%%%%%%%%%%%%%%%%%%%%%%%%%%%
	  	Stanton et al.~\cite{Stanton2014} & Rule-based & Robotic & \cmark & \xmark & \xmark & \xmark & \xmark & \xmark & \xmark & \xmark & \xmark\\
	  	%%%%%%%%%%%%%%%%%%%%%%%%%%%%%%%%%%%%%%%%%%%%%%%%%%%%%%%%%%%%%%
	  	Andrist et al.~\cite{Andrist2015} & Rule-based  & Robotic & \cmark & \xmark & \xmark & \xmark & \cmark & \xmark & \xmark & \xmark & \cmark\\
	  	%%%%%%%%%%%%%%%%%%%%%%%%%%%%%%%%%%%%%%%%%%%%%%%%%%%%%%%%%%%%%%
	  	Mohammad et al.~\cite{Mohammad2010} & \ac{ML}-based  & Robotic & \textbf{?} & \cmark & \xmark & \xmark & \xmark & \xmark & \xmark & \xmark & \xmark \\ 
	  	%%%%%%%%%%%%%%%%%%%%%%%%%%%%%%%%%%%%%%%%%%%%%%%%%%%%%%%%%%%%%%
	  	Huang et al.~\cite{Buschmeier2011} & \ac{ML}-based & Virtual & \xmark & \cmark & \cmark & \cmark & \cmark & \cmark & \cmark & \xmark & \xmark\\
	  	%%%%%%%%%%%%%%%%%%%%%%%%%%%%%%%%%%%%%%%%%%%%%%%%%%%%%%%%%%%%%%
	  	Kok et al.~\cite{Kok2012} & \ac{ML}-based & Virtual &  \xmark & \cmark & \xmark & \xmark & \cmark & \xmark & \xmark & \xmark & \xmark\\
	  	%%%%%%%%%%%%%%%%%%%%%%%%%%%%%%%%%%%%%%%%%%%%%%%%%%%%%%%%%%%%%%
	  	Schröder et al.~\cite{Schroder2012} & \ac{ML}-based & Virtual & \xmark & \cmark & \xmark & \cmark & \cmark & \xmark & \xmark & \cmark & \xmark\\
  		\bottomrule
	\end{tabular}
	\caption{Brief comparison regarding how different virtual agents manage strategies. The systems presented here appear in the same order as in the main body of the text.  \protect\cmark, \protect\xmark \, and \textbf{?}, represents whether the specified strategy is applied, not applied or unclear, respectively.}
	\label{fig:comparison:rapportSystems}
	
\end{table}
\addtolength{\tabcolsep}{-1pt}

Current literature suggests continuing the research on learning social behaviours from \ac{WoZ}~\cite{Sequeira2016, Knox2014, Papangelis2014} studies and use primarily \ac{RL}~\cite{Thomaz2006, Kok2012, Zhao2014, Papangelis2014} classifiers (Section~\ref{subsec:ReinforcementLearning}). This class of algorithms are applicable in rapport as there are sequences of states that will help the agent to know when and how backchannels should be produced in order to build rapport (the reward function). In addition, authors suggest developing solutions capable of adapting current course of actions to the current context of the interaction to improve the quality of virtual agents during interactions~\cite{Kopp2007, Zwiers2011, Reidsma2011, Visser2014}.

\begin{table}[]
	\centering
	\begin{tabular}{|l|c|c|}
		\hline
		\textbf{System}              	& \textbf{Training Source} 	& \textbf{Iterative} \\ \hline
		%%%%%%%%%%%%%%%%%%%%%%%%%%%%%%%%%%%%%%%%%%%%%%%%%%%%%%%%%%%%%%%%%%%%%%%%%%%%%%%%%%%%%%%%%%%%%%%%%
		Mohammad et al.~\cite{Mohammad2010} & Direct Samples (Unsupervised) & Yes \\ \hline
		%%%%%%%%%%%%%%%%%%%%%%%%%%%%%%%%%%%%%%%%%%%%%%%%%%%%%%%%%%%%%%%%%%%%%%%%%%%%%%%%%%%%%%%%%%%%%%%%%
		Virtual Rapport 2.0~\cite{Buschmeier2011} 			& Corpus			& No \\ \hline
		%%%%%%%%%%%%%%%%%%%%%%%%%%%%%%%%%%%%%%%%%%%%%%%%%%%%%%%%%%%%%%%%%%%%%%%%%%%%%%%%%%%%%%%%%%%%%%%%%
		\acf{IPL}~\cite{Kok2012}          				& Corpus \& Subjective evaluation			& Yes \\ \hline
		%%%%%%%%%%%%%%%%%%%%%%%%%%%%%%%%%%%%%%%%%%%%%%%%%%%%%%%%%%%%%%%%%%%%%%%%%%%%%%%%%%%%%%%%%%%%%%%%%
		\acf{SAL}~\cite{Schroder2012} & Corpus \& \ac{WoZ}		& Yes \\ \hline
		%%%%%%%%%%%%%%%%%%%%%%%%%%%%%%%%%%%%%%%%%%%%%%%%%%%%%%%%%%%%%%%%%%%%%%%%%%%%%%%%%%%%%%%%%%%%%%%%%
		Restricted Perception~\ac{WoZ}~\cite{Sequeira2016}  & \ac{WoZ}		& Yes \\ \hline
		
	\end{tabular}
	\caption{Brief comparison of methodologies to learn human social behaviours.}
	\label{fig:comparison:vh:systems}
\end{table}

Most of all, the communication goals of interactions must be considered when developing rapport agents. The context in which the communication partners will interact, the inherent limitations of the virtual agents' perceptions and actions and, most importantly, what kind of emotions and actions we want to elicit from the conversational partner are crucial for the development of such agents. For example, in tutoring applications, mutual gaze plays an important role for increased learning performance \cite{OTTESON1980, SHERWOOD1987, Fry1975}, and in negotiation scenarios not reciprocating negative self-disclosure has ben shown to destroy rapport\cite{Bronstein2012}.

