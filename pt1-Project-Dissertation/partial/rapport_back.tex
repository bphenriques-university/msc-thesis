\subsection{Rapport}
\label{subsec:Rapport}
The feeling of flow and connection during interactions is formally known as rapport~\cite{Wang2009}. It is a phenomenon that affects people on three levels: emotional, behavioural and cognitive~\cite{Wang2009}.

The emotional level refers to the impact the relationship has on partners, while behavioural refers to, for example, the convergence of movements and facial expressions. Finally, the cognitive level refers to a shared understanding between conversational partners~\cite{Wang2009}.

Spender Oatey~\cite{Spencer-Oatey2005} suggests that rapport management can be divided into three main tasks: enhancement, maintaining and destroy. The first task aims to create strong first impressions, the second encourages the continuation and the third task aims to destroy relationships. Each one of these tasks can be modelled as abstract goals that can be accomplished by achieving sub-goals only satisfied by interacting with the external world~\cite{Papangelis2014, Zhao2014} (see Figure~\ref{table:BuildingRapportPlan}). These sub-goals manipulate the three components of rapport suggested by Tickle-Degnen and Rosenthal~\cite{Tickle-Degnen1990}:

\begin{itemize}
	\item \textbf{Positivity}: feeling of approval and friendliness (e.g. head nod and smile);
	\item \textbf{Mutual attention}: feeling that the other's attention is focused on the individual (e.g. mutual gaze and ``hmm hmm'' vocalisations);
	\item \textbf{Coordination}: feeling of predictability and being in-sync (e.g. postural mimicry and synchronised movements).
\end{itemize}

\vspace{-5mm}
\begin{figure}
	\centering
	%\begin{framed}
		\scalebox{0.65}{
			\begin{forest}
				[\textbf{Build Rapport}
				[Stimulate Positivity 
					[Elicit Positive emotions [Smile] [Embarrassed Laugh][Praise]]]
				[Stimulate Mutual Attention
					[Mutual Gaze] [Voice utterances][Paraphrasing]]
				[Stimulate Coordination
					[Postural Mirror]
					[Be predictable
						[Adhere To Social Norms
							[Greet] [Be Polite]]]]
				]
			\end{forest}
		}
	%\end{framed}
	\caption{Example plan for building rapport. The nodes are goals and the leafs are actions.}
	\label{table:BuildingRapportPlan}
\end{figure}
\vspace{-4mm}

However, in order to build rapport, it is essential that these three components co-exist during an interaction~\cite{Grahe1999, Wang2010, Zhao2014, Cassell2007}. For example, using gaze to establish mutual attention conveys disinterest and can have negative social effect if not accompanied by other behaviours that stimulate positivity and coordination~\cite{Wang2010}. Although, the relative weights of these components may change as the relationship evolves beyond strangers~\cite{Wang2010, Zhao2014, Cassell2007}.

Moreover, two strangers behaviours are initially driven by cultural conventions as they do not know each other and, therefore, they expect what was taught by their cultural environment according to the current context~\cite{Zhao2014} (e.g. greet and be polite). As the relationship evolves and the interlocutors get to know one another, positivity decreases and it is replaced by coordination while mutual attention remains constant~\cite{Zhao2014, Tickle-Degnen1990}. The interlocutors may even violate what is culturally accepted in order to meet interactional goals and behavioural expectations~\cite{Zhao2014}. For example, friends may use sarcasm and insults instead of politeness~\cite{Zhao2014}. Table~\ref{table:rapportStrategies} describes examples of verbal and non-verbal strategies for managing rapport.

\vspace{-3mm}
\begin{table}[]
	\centering
	\begin{tabular}{@{}ll@{}}
		\toprule
		\multirow{2}{*}{\textbf{Verbal}}    & Humour; Paraphrasing; Self disclosure; Praise; Ego Suspension.\\ 
		& Refer to shared experience; Slower rate of speech; Small-talk; \\ \midrule
		\multirow{2}{*}{\textbf{Non-verbal}} & Gaze; Smile; Reciprocate previous action. Silence; Postural mimicry; \\  
		& Gesture mimicry; Mirror Facial Expression; Head gestures. \\ \bottomrule
	\end{tabular}
	\caption{Examples of strategies and actions to manage rapport.}
	\label{table:rapportStrategies}
\end{table}
\vspace{-8mm}

Learning behavioural expectation is also important to assess rapport success. This can be achieved through the use of self-disclosure \cite{Moon2000}, small-talk~\cite{Cassell2003} and humour~\cite{Treger2013}. For instance, when self-disclosure is successful, it is reciprocal, intimacy increases, disclosed topics become more diversified, and deeper~\cite{Zhao2014}. Moreover, assessing when rapport strategies were successful is also important, for example, mutual gaze and smiling become more noticeable and consistent\cite{Grahe1999, Zhao2014}.

To sum up, rapport is a phenomenon that makes interactions more engaging and harmonious. Rapport management can be modelled as a problem of goal satisfaction that is solved through the realisation of actions. However, the strategies for managing rapport must take into account the goals of the interaction and the sociocultural context of the interaction in order to satisfy behavioural expectations. Although, these expectations may not be clear at first and strategies for learning them should be applied.