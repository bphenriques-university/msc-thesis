% This is "Introduction.tex"

\begin{figure*}
	\centering
	%\begin{framed}
		\scalebox{0.85}{
            \begin{forest}
                [\textbf{Build Rapport}
                    [Stimulate Positivity 
                        [Self-disclosure][Motivate][Humor]]
                    [Stimulate Mutual Attention
                        [Mutual Gaze][Backchannel]]
                    [Stimulate Coordination
                        [Behavioural Mimicry
                            [Facial Expressions][Head Gestures]]
                        [Adhere To Social Norms]
                        [Backchannel]]
                ]               
            \end{forest}
        }
	%\end{framed}
	\caption{Example of a goal tree to build rapport. The nodes are goals and the leafs are actions.}
	\label{table:BuildingRapportPlan}
\end{figure*}

\section{Introduction}
\label{sec:Introduction}

Robots are increasingly becoming part of our society and their presence has been proven to impact our lives. But do any of us remember a remarkable interaction with a robot to the same degree we are able to recall one with a person? What makes one conversation memorable? 

In order to answer these questions, researchers have been exploring agents capable of building rapport, i.e., designing agents that consider the following aspects during interactions: positivity, mutual attention, and coordination~\cite{Spencer-Oatey2005}. In reality, most of the today's agents do not consider formally this concept, and yet they have been impacting people's lives on several scenarios such as education~\cite{Burroughs2007}, child care~\cite{Burns1984} and medical assistance~\cite{Kang2005, Lisetti2013}. Researchers have been mostly considering single aspects of rapport such as gaze~\cite{Andrist2015, Mutlu2006, Stanton2014, Andrist2014, Baxter2014, Wang2010} and backchannel~\cite{Truong2011, Huang2010, Poppe2011, Poppe2010, Kok2012, Niewiadomski2009} (listener behaviour), which is not enough, as rapport is managed more efficiently when considering its three components. More importantly, we only found virtual agent capable of managing these three components~\cite{Buschmeier2011, Gratch2006}. 

This work tackles the lack of robotic agents capable of building rapport using its three components. For this purpose, we designed and implemented a rapport model (Sections~\ref{sub:rapportModel} and ~\ref{sec:model_implementation}, respectively) on robot \ac{EMYS} using the \ac{SERA} ecosystem~\cite{Tullio2015}. The designed rapport model enables concurrent execution of rapport behaviours using a prioritisation mechanism where idle actions (e.g., postural mimicry) have lower priorities than behaviours triggered momentarily (e.g., surprise animations). These strategies may be either rule-based or \ac{ML}-based but they all cooperate to achieve the interactional goals of the scenario.

To analyse the performance of the developed rapport strategies, we conducted user studies to study its impact on likeability, intelligence and liveness using Godspeed questionnaires~\cite{bartneck2009measurement, lehmann2015good}, and proximity~\cite{aron1992inclusion} using the Quick Numbers scenario detailed in Section~\ref{sec:studies}.