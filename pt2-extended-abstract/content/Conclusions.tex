% This is "Conclusions.tex"
\section{Discussion}

The developed robotic rapport agent did not reveal statistically significant results between the control and the rapport condition, given the subjective measurements collected from the questionnaires. We believe that the lack of statistical significant result was due to the low sample size ($n=20$ in each condition) and due to the fact that the participants were more focused on the task and not on the agent. Notwithstanding, participants showed more positive reactions in the rapport condition. On one occasion, the participant remarked the agent's capability to synchronise his happy animation with its laugh. One another two independent occasions, both male and female participants flattered the agent's ability to distinguish their gender. Furthermore, it is also possible to retrieve behaviour metrics such as smile frequency and eye contact duration, by annotating the recorded video (1235 minutes) to look for additional metrics that may confirm that the rapport model improves current agents.

\section{Conclusion and future work}

In this paper, we presented the design and implementation of a rapport model that enables robotic and virtual agents to show natural signs of rapport on any \ac{HRI} scenario according to the dyadic state of the interaction. The resulting framework was built on top of the \ac{SERA} ecosystem and tested using robot~\ac{EMYS} on the Quick Numbers scenario.

Overall, the system reveals to be a success as we are currently able to reuse the rapport model on any \ac{SERA} agents with low effort. In addition, we made the first steps on building richer agents by integrating the \ac{SSI} and SHORE on our system which increases the agent's perceptual capabilities. Furthermore, we encourage researchers to design rapport agents, as it is noticeable from the experiments that it is possible to bring forth more positive emotions by using dedicated rapport strategies.

In the future, we propose using a sample size greater than 20, use participants with no previous experience with the agent and improve the scenario to provide more quality rapport opportunities. In addition, we should look into behavioural metrics such as eye contact and smile frequency. The rapport model can be improved by considering mechanisms to assess rapport success so that the agent might adapt its strategies to the interaction, and even recover from mistakes~\cite{Kahn2008}. For example, use SHORE to monitor the emotion state of the user and attempt use humour to cheer him as soon as sadness becomes the most average emotion. The model should also explore resumable actions, that is, explore agents capable of resuming their course of action after being interrupted by external stimuli. Finally, the backchannel \textit{Effector} should be revisited as it is only lacking an improved noise suppressing mechanism.